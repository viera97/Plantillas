% !TeX encoding = UTF-8
% !TeX spellcheck = es_ES
\documentclass{article}

\usepackage[utf8]{inputenc}
\usepackage[spanish]{babel}

\topmargin = -1.5cm
%\textwidth  = 18.5cm
\textwidth  = 18.1cm
\textheight  = 22cm
%\oddsidemargin =-1cm
\oddsidemargin =-0.7cm
%\pagestyle{empty}

\usepackage{ragged2e}
\usepackage{footmisc}

\usepackage{xcolor}

\definecolor{uprgreen}{RGB}{18,104,47}
\definecolor{uprgreen2}{RGB}{51,144,55}

\usepackage{hyperref}
\hypersetup{
	colorlinks=true,
	urlcolor=uprgreen2,
	linkcolor=uprgreen2
}
%\usepackage[breaklinks,colorlinks=true,linkcolor=red,citecolor=red, urlcolor=blue]{hyperref}

\usepackage{amsmath,amssymb,amsfonts,latexsym, graphicx}

\newcommand{\foot}{\fontsize{10}{13}\selectfont}
\newcommand{\notfoot}{\justifying\fontsize{13}{18}\selectfont}

\begin{document}

	{\bf\Large\color{uprgreen} Gestión de los Procesos Universitarios}
\\
	
	{\color{uprgreen}\hrule}
	\begin{flushleft}
		\begin{minipage}{7cm}
			{\fontsize{13}{18}\selectfont
			Dayron Viera Quintero
			}
		\end{minipage}
		\hfill
		\begin{minipage}{7cm}
			\begin{flushright}
				{\fontsize{13}{18}\selectfont
				\href{mailto:dayron.vieraq@upr.edu.cu}{dayron.vieraq@upr.edu.cu}
				}
			\end{flushright}
		\end{minipage}
	\end{flushleft}
	{\color{uprgreen}\hrule}
	\section*{Tarea 4.}
	
	\notfoot
	
	Aunque originalmente el liderazgo surgió en el ámbito empresarial, donde los factores que se interrelacionan están bien definidas e identificadas, este se ha expandido debido a la necesidad a otros ámbitos como el educacional, lo cual es un medio incluso más complejo que el ámbito empresarial y político. \cite{educ}
	
	El líder es un intelectual que posee una visión crítica, habilidades de gestión y conocimientos necesarios para crear y facilitar espacios para la participación y el cambio, pero el líder educacional debe además ser capaz de transformar, de ser necesario, las formas en las cuales la escolarización, las políticas educativas y el aprendizaje están conformadas.\cite{PGL}\cite{BLE}
	
	Entre las principales problemáticas que puede enfrentar un líder educacional está la ausencia de facultades para gestionar ciertos procesos y administrar ciertos recursos tanto materiales como humanos, esto entorpece grandemente los procesos de liderazgo sobre todo con un líder de ``bajo rango''. El líder en el contexto cubano, con nuestra cultura tan peculiar, debe tener una presencia que imponga respeto y confianza, tiene que tener grandes habilidades de oratoria para poder motivarlos, dentro de esto entra varios factores como expresiones corporales, tono de voz entre otros.
	
	El líder educacional tiene la problemática de representar a muchas personas que incluso pueden tener opiniones contrarias, por lo que debe ser capaz de hacer llegar a un acuerdo o de analizar con total imparcialidad la situación así apoyar lo que crea correcto. Entre las mayores problemáticas se encuentra la gestión del tiempo en las distintas áreas de gestión, lo cual es muy complicado sobre todo en un entorno tan diverso como lo es el ámbito educacional.
	
	El líder debe ser capaz de anteponerse a todos estos problemas y a los que se le presente y así formar una robusta confianza por parte de los ``liderados''.

	\bibliographystyle{plain}
	\bibliography{refs}

\end{document}
