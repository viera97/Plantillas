% !TeX encoding = UTF-8
% !TeX spellcheck = es_ES
\documentclass{article}

\usepackage[utf8]{inputenc}
\usepackage[spanish]{babel}

\topmargin = -1.5cm

%%\textwidth  = 18.5cm

\textwidth  = 18.1cm
\textheight  = 22cm

%%\oddsidemargin =-1cm

\oddsidemargin =-0.7cm

%%\pagestyle{empty}

\usepackage{ragged2e}
\usepackage{footmisc}

\usepackage{xcolor}

%\usepackage[breaklinks,colorlinks=true,linkcolor=red,citecolor=red, urlcolor=blue]{hyperref}

\usepackage{amsmath,amssymb,amsfonts,latexsym, graphicx}

%\newcommand{\foot}{\fontsize{10}{13}\selectfont}
%\newcommand{\notfoot}{\justifying\fontsize{13}{18}\selectfont}

\definecolor{uprgreen}{RGB}{18,104,47}
\definecolor{uprgreen2}{RGB}{51,144,55}

\usepackage{hyperref}
\hypersetup{
	colorlinks=true,
	urlcolor=uprgreen2,
	linkcolor=uprgreen2
}

\begin{document}
$$
\begin{array}{c}
	\text{\color{uprgreen}\bf\Huge Gestión de Procesos}\\ \\
	\text{\color{uprgreen}\bf\Huge Universitarios}
\end{array}
$$

\begin{figure}[h!]
	\centering
	\includegraphics[scale=0.2]{Identidad-a-color-884x1024.png}
\end{figure}

{\large {\it Lic}. Dayron Viera Quintero \hfill \href{mailto:dayron.vieraq@upr.edu.cu}{dayron.vieraq@upr.edu.cu}}\\ \\ \\ \\

{\color{uprgreen}\bf\huge Índice:}
\begin{itemize}
	\item \hyperref[sec:Intro]{\large Introducción.}
	\item \hyperref[sec:EME]{\large Elección del Método de Evaluación.}
	\item \hyperref[sec:MGU]{\large Modelos de Gestión Universitarios.}
	\item \hyperref[sec:PDG]{\large Problemas en las Dimensiones de mi Gestión.}
	\item \hyperref[sec:PRPLDU]{\large Problemáticas y Retos que Poseen los Líderes Docentes Universitarios.}
	\item \hyperref[sec:CON]{\large Conclusiones.}
	\item \hyperref[sec:REF]{\large Referencias.}
\end{itemize}

\newpage
\section*{\color{uprgreen} Introducción.}
\label{sec:Intro}
{\justifying\fontsize{13}{18}\large

Los temas de discusión acerca de la calidad de la educación superior son recurrentes en la actualidad. Con la expansión de la educación superior, en conjunto con la globalización, han hecho repensar el modo de producción de enseñanza, y si es el más adecuado para los tiempos actuales. Sin duda, que cada universidad dedica recursos para cada una de sus procesos, la docencia, la investigación y la extensión universitaria. Dichos recursos, por lo general son escasos y, por tanto, la organización se ve obligada a diversificar sus ingresos. La política de diversificación universitaria queda determinada por los papeles sociales como personal administrativo y de apoyo y una población variable de estudiantes de muy diverso tipo y condición. De ahí la importancia de una eficiente y competente gestión de los procesos universitarios.

En este portafolio serán expuestos diversos temas tratado a lo largo del curso de Gestión de Procesos Universitarios de la Universidad de Pinar del Río. Se exponen criterios sobre la importancia de la gestión educativa, los problemas en las dimensiones, así como los retos que presentan los líderes docentes universitarios.
}

\section*{\color{uprgreen} Elección del Método de Evaluación.}
\label{sec:EME}

{\justifying\fontsize{13}{18}\large Tanto el portafolio como el proyecto de investigación son buenas muestras del conocimiento obte-
nido por el estudiante durante el curso, son necesarias para su confección las bases de la asignatura,
y la capacidad de razonar sobre estos contenidos.

El proyecto investigativo sería muy buena opción como ejercicio de culminación, puesto que en
este se debe ampliar y estudiar más a fondo los temas que aborde el mismo. Con el proyecto de
investigación se pudieran obtener resultados útiles para otras investigaciones y para su uso por otros
profesionales, pudiera ser el pie de inicio a una futura publicación que contribuya a nuestro curriculum
y formación como futuro docente; sin embargo para realizar el proyecto es necesario conocimientos
de varios temas de la asignatura, y debido a la brevedad del curso se hace algo complicado completar
un proyecto investigativo en tan poco tiempo.

El portafolio también sería muy buena opción, es una forma integradora y atractiva de medir los
conocimientos y el progreso del estudiante, es también forma de que el propio estudiante mida su
conocimientos sobre la asignatura. Es una compilación de contenidos, resultados, tareas realizadas,
entre otros, por lo que es muy útil para realizar consultas sobre los temas estudiados y trabajados en
caso de necesitarlo. Cabe destacar sobre el portafolio que debido a su modo de confección es posible ir
confeccionandolo sobre la marcha por lo que sería mucho más fácil organizar el tiempo de realización.
Por todo lo antes expuesto y viendo algunas de las ventajas que consideré del portafolio sobre el
proyecto investigativo, he decidido como tarea integradora final del curso la confección del portafolio.}

\section*{\color{uprgreen} Modelos de Gestión Universitarios.}
\label{sec:MGU}

{\justifying\fontsize{13}{18}\large La gestión educativa es de suma importancia en un centro como el nuestro, de enseñanza superior,
en dependencia de la situación y condición del centro se pueden tomar diferentes enfoques acorde a
las necesidades. Los enfoques son llevados a la práctica a través de modelos que responden de manera
concreta a condiciones concretas.

Entre los principales modelos de gestión podemos mencionar, el modelo Normativo que estuvo
vigente entre los años 50 y 60, el modelo Prospectivo, el cual se comenzó a materializar a partir de
los años 70, el Estratégico a partir de los años 80 junto con el Estratégico Situacional, luego surge el
modelo de Calidad Total en los años 90, el de Reingeniería, y finalmente el Comunicacional. \hyperref[RUP]{\text{$[4]$}}

Los modelos de gestión más acorde a mi concepción son el modelo Prospectivo y el modelo de
Reingeniería. El modelo Prospectivo se basa en la planificación a futuro, con múltiples escenarios de
posibles futuros, haciendo que la educación se vaya modificando a largo plazo para adaptarse a futuras
necesidades. Este modelo aporta una muy buena idea para un posible enfoque de la educación, sin
embargo prefiero el modelo de Reingeniería. Este modelo surge a mediados de los 90 y se basa en la
planificación sobre contextos cambiantes, es capaz de adaptarse en cortos períodos de tiempos a las
necesidades situacionales de cada entidad, proporciona una mayor calidad y exigencia en la educación.
A pesar de su futuro incierto este modelo es capaz de mantener ``al día'', la producción de recursos
humanos capacitados para nacientes campos y especialidades.

En caso de ponerlo en práctica daría mucho beneficios por lo ``actualizados'' que se pudiera estar en
materia de especialidades, sin embargo esto mismo puede traer malas consecuencias, ya que pudiera
gastarse recursos y tiempo en posibles campos que nunca den fruto, por lo que para ello se necesita
investigaciones profundas y diarias de las nuevas posibilidades y su posible desarrollo.
El modelo de gestión que puedo percibir en mi área en la Universidad de Pinar del Río es el modelo
Prospectivo y el Comunicacional, como ya se mencionó anteriormente el Prospectivo está orientado de
cara a múltiples opciones de futuro; los programas y enfoques de las carreras van cambiando acorde a
las posibles necesidades futuras, y en cuanto al Comunicacional, este se evidencia a la hora de tomar
decisiones y posibles acciones, ya que se comparte la responsabilidad y se conforman equipos para el
análisis de estas ciertas decisiones.}

\section*{\color{uprgreen} Problemas en las Dimensiones de mi Gestión.}
\label{sec:PDG}
{\justifying\fontsize{13}{18}\large 
Entre las proyecciones de la gestión educativa se pueden distinguir diferentes acciones y agruparse según su naturaleza, así podemos discernir entre acciones de tipo pedagógica, administrativa, institucional y comunitaria; estos son los distintos planos de acciones que se complementan para el funcionamiento de la gestión. \hyperref[RUP]{\text{$[4]$}}

Todas estas dimensiones son importantes y se ocupan de procesos necesarios para el buen funcionamiento de una institución pedagógica como lo es nuestra universidad. Entre las mayores preocupaciones que encuentro está dentro de la dimensión institucional los canales de comunicación, debido a que el principal canal de comunicación es el correo institucional, y por diversos motivos tiene un funcionamiento poco estable, haciendo que se interrumpa el acceso a otras plataformas como el aula virtual, Moodle. También es necesario hacer ``promoción'' de otros medios de comunicación como es el canal de \href{https://t.me/CanalUprCuba}{telegram} de la UPR, el cual a pesar de ser un buen medio de comunicación muy completo y actualizado es poco conocido, con un número de suscriptores bastante bajo para una institución universitaria, y con la misma cuestión se encuentra la cuenta oficial de la universidad en \href{https://twitter.com/UPR_Cuba}{twitter}.

En la dimensión pedagógica en mi opinión existe otro problema importante y es que los estudiantes realizan muy pocos trabajos investigativos, por lo que es necesario fomentar las investigaciones y proyectos, y mejorar la calidad de los mismos, que tengan un mayor impacto en problemas de la sociedad. Esto también se relaciona con la proyección social en la dimensión comunitaria, debido a que es necesario una mayor interacción entre empresas e instituciones con la universidad para proponer proyectos relacionados con la formación del profesional.

La dimensión que siento que más me preocupa es la dimensión pedagógica, es necesario motivar a los estudiantes a utilizar, en el caso de mi departamento, la matemática en problemas aplicados a sus formaciones. Por tal motivo incorporaría a mi gestión como docente un tiempo dedicado únicamente a formular posibles proyectos para los estudiantes, y servir de tutor en la medida de lo posible acorde a mi posición de adiestrado. Para esto es necesario el empleo de mi tiempo y la ayuda de profesores de mayor experiencia, así como la bibliografía necesaria para cada tema.

Dado los expuesto anteriormente plantearía posibles soluciones a las preocupantes, como fortalecer la estabilidad de las plataformas de comunicación y hacer llegar los medios de comunicación alternativos como la cuenta de \href{https://twitter.com/UPR_Cuba}{twitter} y \href{https://t.me/CanalUprCuba}{telegram} de la UPR. En cuanto a la dimensión pedagógica puedo proponer como solución lo que se explicó en el párrafo anterior, unido al diálogo con las empresas e instituciones fuera de la universidad para la realización de proyectos. Todas estas acciones se pueden realizar este año, su implementación no sería un gran problema, la mayor dificultad pudiera ser la interacción con las instituciones externas pero creo que sería posible con una buena gestión de la dimensión comunitaria.}

\section*{\color{uprgreen} Problemáticas y Retos que Poseen los Líderes Docentes Universitarios.}
\label{sec:PRPLDU}
{\justifying\fontsize{13}{18}\large

Aunque originalmente el liderazgo surgió en el ámbito empresarial, donde los factores que se interrelacionan están bien definidas e identificadas, este se ha expandido debido a la necesidad a otros ámbitos como el educacional, lo cual es un medio incluso más complejo que el ámbito empresarial y político. \hyperref[educ]{\text{$[3]$}}

El líder es un intelectual que posee una visión crítica, habilidades de gestión y conocimientos necesarios para crear y facilitar espacios para la participación y el cambio, pero el líder educacional debe además ser capaz de transformar, de ser necesario, las formas en las cuales la escolarización, las políticas educativas y el aprendizaje están conformadas. \hyperref[PGL]{\text{$[1]$}}\hyperref[BLE]{\text{$[2]$}}

Entre las principales problemáticas que puede enfrentar un líder educacional está la ausencia de facultades para gestionar ciertos procesos y administrar ciertos recursos tanto materiales como humanos, esto entorpece grandemente los procesos de liderazgo sobre todo con un líder de ``bajo rango''. El líder en el contexto cubano, con nuestra cultura tan peculiar, debe tener una presencia que imponga respeto y confianza, tiene que tener grandes habilidades de oratoria para poder motivarlos, dentro de esto entra varios factores como expresiones corporales, tono de voz entre otros.

El líder educacional tiene la problemática de representar a muchas personas que incluso pueden tener opiniones contrarias, por lo que debe ser capaz de hacer llegar a un acuerdo o de analizar con total imparcialidad la situación así apoyar lo que crea correcto. Entre las mayores problemáticas se encuentra la gestión del tiempo en las distintas áreas de gestión, lo cual es muy complicado sobre todo en un entorno tan diverso como lo es el ámbito educacional.

El líder debe ser capaz de anteponerse a todos estos problemas y a los que se le presente y así formar una robusta confianza por parte de los ``liderados''.
}

\section*{\color{uprgreen} Conclusiones.}
\label{sec:CON}
{\justifying\fontsize{13}{18}\large En conclusión podemos decir que en este portafolio se han expuesto diversas problemáticas y posibles soluciones dentro del campo de la gestión universitaria.} Este portafolio sirve de base para posibles estudios en la rama y sirve como una constancia de los contenidos y temas tratados durante el curso de Gestión de Procesos Universitarios del curso de Diplomado de la Universidad de Pinar del Río. 

\section*{\color{uprgreen} Referencias}
\label{sec:REF}
\begin{enumerate}
	\item[\text{$[1]$}]\label{PGL} {\it Problemas de gestión asociados al liderazgo como función directiva.} \href{https://www.scielo.cl/scielo.php?pid=S0718-07052012000100007\&script=sci\_arttext}{https://www.scielo.cl/scielo.php?pid=S0718-07052012000100007\&script=sci\_arttext}, 2022. Consultado en 31 octubre, 2022.
	\item [\text{$[2]$}]\label{BLE} Fernando Gaspar Alfredo Rojas. {\it Bases del Liderazgo en Educación}. 2006.
	\item [\text{$[3]$}]\label{educ} José Antonio Pareja Fernández de la Reguera. Liderazgo y conflicto en las organizaciones educativas. {\it educ.educ.}, 12\(1\), 2009.
	\item [\text{$[4]$}]\label{RUP} Representación de la UNESCO en Perú. {\it Manual de Gestión para Directores de Instituciones Educativas.} 2011.
\end{enumerate}

\end{document}
